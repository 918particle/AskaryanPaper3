\documentclass[amsmath,amssymb,aps,prd,10pt,twocolumn,showkeys]{revtex4}
\usepackage{graphicx}
\usepackage{mathtools}
\usepackage{verbatim}
\DeclareMathOperator\erfc{erfc}
\DeclareMathOperator\erf{erf}
\DeclareMathOperator{\sgn}{sgn}
\DeclareMathOperator{\snr}{SNR}
\begin{document}

\title{Complex Analysis of Askaryan Radiation: UHECR Reconstruction with Askaryan Radio Array}

\author{Jordan C. Hanson}
\email{jhanson2@whittier.edu}
\affiliation{Department of Physics and Astronomy, Whittier College}
\author{Damian Iba\~{n}ez-Rodriguez}
\affiliation{Department of Physics and Astronomy, Whittier College}
\date{\today}

\begin{abstract}
Ultra-high energy cosmic rays (UHECR) can produce relativistic cascades that emit radio-frequency (RF) pulses in the 10-1000 MHz bandwidth via two distinct effects: the geomagnetic effect, and the Askaryan effect.  The geomagnetic effect occurs when the magnetic field of the Earth causes cascade charges to form a transverse current that radiates linearly polarized radiation aligned with the Lorentz force direction.  The Askaryan effect is caused by the net negative charge excess in the cascade that radiates linearly polarized radiation along the Cherenkov cone.  When UHECR cascades enter solid, RF transparent matter at altitudes where the cascade develops, Askaryan radiation can propagate through the solid matter to RF detectors.  The Askaryan Radio Array (ARA) at the South Pole has observed 13 UHECR candidates in precisely this fashion.  Recently, the ARA collaboration published evidence of 13 UHECR candidates.  We present an analytical model that confirms the events are UHECRs.  The model includes the Askaryan effect and the ARA RF channel response.  The coherently summed waveforms (CSWs) from the UHECR candidates match our model with correlation coefficients greater than 0.8, and with minimal fractional power differences.  From the fit between model and data, we extract the original electromagnetic Askaryan pulses.
\end{abstract}

\keywords{Ultra-high energy neutrino; Askaryan radiation; Mathematical physics}

\maketitle

\section{Introduction}
\label{sec:int}

\begin{enumerate}
\item Define UHECRs
\item Brief summary of radio detection of UHECRs
\begin{itemize}
\item Geomagnetic effect
\item Askaryan effect
\end{itemize}
\item Why Askaryan effect can dominate UHECR signals in ARA
\begin{itemize}
\item Cite recent paper
\item When cascade interacts directly with ice
\item Do models exist for this?  Likely not
\item Gives us a measurement of the E-field
\end{itemize}
\item This paper is organized as follows...
\end{enumerate}

\section{Units, Definitions, and Conventions}

\begin{enumerate}
\item Define $s(t)$, and parameters $E_0$ and $\sigma_t$
\item Define $r(t)$, and parameters $f_0$ and $\gamma$
\item Define convolution $s * r$
\item Define cross-correlation $d \star x$
\item Define CSW
\end{enumerate}

\section{The Askaryan Radio Array}

\begin{enumerate}
\item Detector diagram of ARA
\item Event diagram of UHECR interaction
\item Cite recent detection of UHECR events
\item Calibration of $f_0$ from data itself
\item Calibration of $\gamma$ from data itself
\end{enumerate}

\section{Reconstruction Analysis}

\begin{enumerate}
\item Waveform reconstruction
\begin{itemize}
\item Present graphs and tabulated results for CSWs by event
\item Present graphs and tabulated results for subsets of channels by event
\end{itemize}
\item E-field calculations
\begin{itemize}
\item E-field of UHECRs from CSWs
\item E-field of UHECRs from channel subset CSWs
\end{itemize}
\end{enumerate}

\begin{figure*}
\centering
\includegraphics[width=0.49\textwidth]{figures/1915_26288.pdf}
\includegraphics[width=0.49\textwidth]{figures/1957_13330.pdf}
\includegraphics[width=0.49\textwidth]{figures/2171_31805.pdf}
\includegraphics[width=0.49\textwidth]{figures/2250_20189.pdf}
\caption{\label{fig:1} UHECRs, by ID: (top left) 1915-26288, (top right) 1957-13330, (bottom left) 2171-31805, (bottom right) 2250-20189.}
\end{figure*}

\begin{figure*}
\centering
\includegraphics[width=0.49\textwidth]{figures/2352_85489.pdf}
\includegraphics[width=0.49\textwidth]{figures/2375_17342.pdf}
\includegraphics[width=0.49\textwidth]{figures/2529_09767.pdf}
\includegraphics[width=0.49\textwidth]{figures/2716_58611.pdf}
\caption{\label{fig:2} UHECRs, by ID: (top left) 2352-85489, (top right) 2375-17342, (bottom left) 2529-09767, (bottom right) 2716-58611.}
\end{figure*}

\begin{figure*}
\centering
\includegraphics[width=0.49\textwidth]{figures/3352_89556.pdf}
\caption{\label{fig:3} UHECRs, by ID: 3352-89556}
\end{figure*}

\begin{table}
\centering
\begin{tabular}{| c | c | c | c | c | c |}
\hline
\textbf{ID} & \textbf{$\sigma_{t,1}$ (ns)} & \textbf{$\sigma_{t,2}$ (ns)} & \textbf{$\Delta t$ (ns)} & \textbf{Rel. amp.} & \textbf{$\rho$} \\
1915-26288 & 0.6 & 2.0 & 25.25 & 0.08 & 0.86 \\
1957-13330 & 0.5 & 1.8 & 11.5 & 0.09 & 0.74 \\
2171-31805 & 0.6 & 1.9 & 25.25 & 0.09 & 0.81 \\
2250-20189 & 0.7 & 2.2 & 24.75 & 0.09 & 0.81 \\
2352-85489 & & & & & \\
2375-17342 & & & & & \\
2529-09767 & & & & & \\
2716-58611 & & & & & \\
2782-00106 & & & & & \\
2955-47449 & & & & & \\
2961-98361 & & & & & \\
2978-29412 & & & & & \\
3352-89556 & & & & & \\
\hline
\end{tabular}
\end{table}

\section{Conclusion}
\label{sec:conc}

What did we learn?  How to reconstruct the E-field, and the product $\sigma_t \propto a\Delta\theta$ tells us about the event geometry and energy

\appendix

\section{The appendix}
\label{sec:app}

Derivation of the $s * r$, maybe the code to do it

\bibliography{apssamp}

\end{document}

